\section{Ágensek vezérlése}
A model elkészítéséhez a következő lépéseket tettem meg:
Data Preprocessing
Feature Engineering
Class imbalance megoldás, mivel a megadott adatbázisban.


\textbf{Hivatkozásra példa}

Az ágensek vezérléséhez a potenciálmező navigációs módszer volt felhasználva. Ez egy bevált módszer a robotrajok vezérléséhez \cite{szanto2015investigation}. Az alapötlete, hogy az akadályok taszító erővel hatnak az ágensre és a cél vonzó erővel. Ennek a két erőnek az eredője határozza meg az irányt amerre érdemes haladni.

\subsection{Potenciálmező navigáció}

\textbf{Egyenletekre példa}

A potenciálmező navigációs módszernél az erők nagysága az \eqref{eq:poti} egyenlet szerint van kiszámolva.

\begin{equation}
    \left\{
    \begin{array}{l}
        |\vec{f}_{push}| = a e ^ {- \frac{(x - b_{push}) ^ {2}}{2 c_{push}^2 }} \\
        |\vec{f}_{pull}| = a e ^ {- \frac{(x - b_{pull}) ^ {2}}{2 c_{pull}^2 }} \\
    \end{array}
    \right.
    \label{eq:poti}
\end{equation}

\begin{itemize}
    \item a: Gauss görbe magassága
    \item b: Gauss görbe középpontja
    \item c: Gauss görbe szélessége
\end{itemize}

\begin{equation}
    \vec{f}_{robot} = \sum_{i} \vec{f}_{push_{i}} + \sum_{i} \vec{f}_{pull_{i}}
    \label{eq:poti_eredo}
\end{equation}

Az eredő vektor a \eqref{eq:poti_eredo} képlet szerint volt kiszámolva. 
